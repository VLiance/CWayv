\chapter{Types}
\label{types}

The C\textasciitilde\, type system knows seven type groups:

 
\begin{center}
\begin{tabular}{|l | l | l|}
	\hline
	Group &  Type &  Prefix \\ \hline
	Numeric & Int, Float & n \\
	String & String & s \\
	Array & Array, QueueArray, Map & a \\
	Object & Class Object & o \\ 
	Function & Function & f \\
	Delegate & Delegate & d \\
	Postprocessing & Typedef & t \\ \hline
	
\end{tabular}
\end{center}


\section{Basic Types}
\label{types-basic-types}

\emph{Basic types} are \type{Bool}, \type{Float} and \type{Int}. They can easily be identified in the syntax by values such as

\begin{center}
\begin{tabular}{| l | l | l | l | l |}
	\hline
	\multicolumn{5}{|c|}{Arithmetic} \\ \hline
	Operator & Operation & Operand 1 & Operand 2 & Return \\ \hline
	\expr{++}& increment & \type{Int} & N/A & \type{Int}\\
	& & \type{Float} & N/A & \type{Float}\\
	\expr{--} & decrement & \type{Int} & N/A & \type{Int}\\
	& & \type{Float} & N/A & \type{Float}\\
	\expr{+} & addition & \type{Float} & \type{Float} & \type{Float} \\
	& & \type{Float} & \type{Int} & \type{Float} \\
	& & \type{Int} & \type{Float} & \type{Float} \\
	& & \type{Int} & \type{Int} & \type{Int} \\
	\expr{-} & subtraction & \type{Float} & \type{Float} & \type{Float} \\
	& & \type{Float} & \type{Int} & \type{Float} \\
	& & \type{Int} & \type{Float} & \type{Float} \\
	& & \type{Int} & \type{Int} & \type{Int} \\
	\expr{*} & multiplication & \type{Float} & \type{Float} & \type{Float} \\
	& & \type{Float} & \type{Int} & \type{Float} \\
	& & \type{Int} & \type{Float} & \type{Float} \\
	& & \type{Int} & \type{Int} & \type{Int} \\	
	\expr{/} & division & \type{Float} & \type{Float} & \type{Float} \\
	& & \type{Float} & \type{Int} & \type{Float} \\
	& & \type{Int} & \type{Float} & \type{Float} \\
	& & \type{Int} & \type{Int} & \type{Float} \\
	\expr{\%} & modulo & \type{Float} & \type{Float} & \type{Float} \\
	& & \type{Float} & \type{Int} & \type{Float} \\
	& & \type{Int} & \type{Float} & \type{Float} \\
	& & \type{Int} & \type{Int} & \type{Int} \\	 \hline
	\multicolumn{5}{|c|}{Comparison} \\ \hline
	Operator & Operation & Operand 1 & Operand 2 & Return \\ \hline
	\expr{==} & equal & \type{Float/Int} & \type{Float/Int} & \type{Bool} \\
	\expr{!=} & not equal & \type{Float/Int} & \type{Float/Int} & \type{Bool} \\
	\expr{<} & less than & \type{Float/Int} & \type{Float/Int} & \type{Bool} \\
	\expr{<=} & less than or equal & \type{Float/Int} & \type{Float/Int} & \type{Bool} \\
	\expr{>} & greater than & \type{Float/Int} & \type{Float/Int} & \type{Bool} \\
	\expr{>=} & great than or equal & \type{Float/Int} & \type{Float/Int} & \type{Bool} \\ \hline
	\multicolumn{5}{|c|}{Bitwise} \\ \hline
	Operator & Operation & Operand 1 & Operand 2 & Return \\ \hline
	\expr{\textasciitilde} & bitwise negation & \type{Int} & N/A & \type{Int} \\	
	\expr{\&} & bitwise and & \type{Int} & \type{Int} & \type{Int} \\	
	\expr{|} & bitwise or & \type{Int} & \type{Int} & \type{Int} \\	
	\expr{\^} & bitwise xor & \type{Int} & \type{Int} & \type{Int} \\	
	\expr{<<} & shift left & \type{Int} & \type{Int} & \type{Int} \\
	\expr{>>} & shift right & \type{Int} & \type{Int} & \type{Int} \\
	\expr{>>>} & unsigned shift right & \type{Int} & \type{Int} & \type{Int} \\ \hline
\end{tabular}
\end{center}

\paragraph{Equality}

\emph{For enums:}
\begin{description}
	\item[Enum without parameters] Are always represent the same value, so \expr{MyEnum.A == MyEnum.A}. 
	\item[Enum with parameters] Can be compared with \expr{a.equals(b)} (which is a short for \expr{Type.enumEquals()}).
\end{description}

\emph{Dynamic:}
Comparison involving at least one Dynamic value is unspecifed and platform-specific.

\subsection{Bool}
\label{types-bool}

\define[Type]{Bool}{define-bool}{Represents a value which can be either \emph{true} or \emph{false}.}

Values of type \type{Bool} are a common occurrence in \emph{conditions} such as \tref{\expr{if}}{expression-if} and \tref{\expr{while}}{expression-while}. The following \emph{operators} accept and return \type{Bool} values:

\begin{itemize}
	\item \expr{\&\&} (and)
	\item \expr{||} (or)
	\item \expr{!} (not)
\end{itemize}

Haxe guarantees that compound boolean expressions are evaluated from left to right and only as far as necessary at run-time. For instance, an expression like \expr{A \&\& B} will evaluate \expr{A} first and evaluate \expr{B} only if the evaluation of \expr{A} yielded \expr{true}. Likewise, the expressions \expr{A || B} will not evaluate \expr{B} if the evaluation of \expr{A} yielded \expr{true}, because the value of \expr{B} is irrelevant in that case. This is important in cases such as this:

\begin{lstlisting}
if (object != null && object.field == 1) { }
\end{lstlisting}

Accessing \expr{object.field} if \expr{object} is \expr{null} would lead to a run-time error, but the check for \expr{object != null} guards against it.




\subsection{Void}
\label{types-void}

\define[Type]{Void}{define-void}{Void denotes the absence of a type. It is used to express that something (usually a function) has no value.}

\type{Void} is a special case in the type system because it is not actually a type. It is used to express the absence of a type, which applies mostly to function arguments and return types.
We have already ``seen'' Void in the initial ``Hello World'' example:
\todo{please review, doubled content}

\haxe{assets/HelloWorld.cw}

The function type will be explored in detail in the section \Fullref{types-function} but a quick preview helps here: The type of the function \expr{main} in the example above is \type{Void->Void}, which reads as ``it has no arguments and returns nothing''.
Haxe does not allow fields and variables of type \type{Void} and will complain if an attempt at declaring such is made:
\todo{review please, sounds weird}

\begin{lstlisting}
// Arguments and variables of type Void are not allowed
var x:Void;
\end{lstlisting}



\section{Nullability}
\label{types-nullability}

\define{nullable}{define-nullable}{A type in Haxe is considered \emph{nullable} if \expr{null} is a valid value for it.}

It is common for programming languages to have a single, clean definition for nullability. However, Haxe has to find a compromise in this regard due to the nature of Haxe's target languages: While some of them allow and; in fact, default to \expr{null} for anything, others do not even allow \expr{null} for certain types. This necessitates the distinction of two types of target languages:

\define{Static target}{define-static-target}{Static targets employ their own type system where \expr{null} is not a valid value for basic types. This is true for the \target{Flash}, \target{C++}, \target{Java} and \target{C\#} targets.}

\define{Dynamic target}{define-dynamic-target}{Dynamic targets are more lenient with their types and allow \expr{null} values for basic types. This applies to the \target{JavaScript}, \target{PHP}, \target{Neko} and \target{Flash 6-8} targets.}

There is nothing to worry about when working with \expr{null} on dynamic targets; however, static ones may require some thought. For starters, basic types are initialized to their default values.
\todo{for starters...please review}

\define{Default values}{define-default-value}{
	Basic types have the following default values on static targets:
	\begin{description}
		\item[\type{Int}:] \expr{0}
		\item[\type{Float}:] \expr{NaN} on \target{Flash}, \expr{0.0} on other static targets
		\item[\type{Bool}:] \expr{false}
	\end{description}
}

As a consequence, the Compiler does not allow the assignment of \expr{null} to a basic type on static targets. In order to achieve this, the basic type has to be wrapped as \type{Null$<$T$>$}:

