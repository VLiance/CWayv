\chapter{Introduction}
\label{introduction}
\state{NoContent}

\section{ What is C·Wayv?}
\label{introduction-what-is-C-Wayv}

 C·Wayv consists of a high-level programming language, inspired of Action Script \& Haxe like language, usually symbolized with a "Wave arrow" (C↝) as title or simply with a tilde (C\textasciitilde) \\


Based on C++, acting like an overlay, it's possible to mix C\textasciitilde/C++. You can also interact with GLSL and Javascript, in a single file, variables from different languages are inter-accessible.\\

C\textasciitilde\, is strongly typed to have robust code and best performances. Main goal is to easily have a single code-base which compiles to multiple targets.\\

It use the powerful \href{https://github.com/VLiance/Cwc}{Cwc} intelligent compiler, which achieve outstanding performances using highly optimized C++ compiler in backend. 

Also this provide the ability to generate any platform output binary since a multitude of backend toolchain can be selected.

Currently, there is the available \href{https://github.com/VLianceTool}{toolchains} :
 
\begin{center}
\begin{tabular}{|l | l | l|}
	\hline
	Toolchain &  From &  Target \\ \hline
	\href{https://github.com/VLianceTool/LibRT}{LibRT}  & Windows & Windows, Linux \\
	libRT(Debian) & Linux & Linux, Windows \\ \hline
\end{tabular}
\end{center}


Haxe abstracts away many target differences, but sometimes it is important to interact with a target directly, which is the subject of \Fullref{target-details}.


\section{Getting Started}
\label{introduction-getting-started}

The following program in C\textasciitilde\, prints ``Hello World'' after being compiled and run:

\haxe{assets/HelloWorld.cw}

This can be easily tested  by saving the above code to a file named \ic{Main.cw} and invoking the Cwc Compiler \\

\subsection{Required tools}
\label{required-tools}

Compilation process is greatly simplified, Cwc is the unique tool you need to build your code. 

Grab the last version of \href{https://github.com/VLiance/Cwc/releases}{Cwc compiler}   \\

Unzip archive, and run Cwc \\

\subsection{First use of Cwc}
\label{use-cwc}

First Cwc ask for:\\
- Demos: Feel free to learn from example. \\
See download section\\
\\
- Setting the Cwc path to your environement. This is usefull when you need to access to Cwc from anywhere in Command-line\\



